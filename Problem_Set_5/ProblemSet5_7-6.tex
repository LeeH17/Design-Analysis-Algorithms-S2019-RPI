\documentclass[11pt, oneside]{article}   	% use "amsart" instead of "article" for AMSLaTeX format
\usepackage{amsfonts}
\usepackage{amsthm}
\usepackage{amsmath}
%\usepackage{setspace}
%\usepackage{graphicx}
%\usepackage{mathtools}
%\usepackage{stackengine}

\newtheorem{Question}{Question}
\newtheorem{Algorithm}{Algorithm}
\newtheorem{claim}{claim}
%\graphicspath{ {} } %\includegraphics{name}

\usepackage{geometry}
\geometry{letterpaper, portrait, margin=1in}

\title {Design and Analysis of Algorithms Assignment 4}
\author{Harrison Lee, Alex Zhao}
\date{February 22, 2019}

\begin{document}

\maketitle

\begin{Question} (7.6) Lightswitch Mapping \\
\end{Question}
\noindent We are given $n$ light fixtures and $n$ light switches in a house. We are also given a subroutine that determines in $O(1)$ time whether two lines intersect.\\
We can use this subroutine to determine in $O(1)$ time if a wall is blocking a fixture and switch pair by inputting the wall as one line segment and the fixture and switchs' coordinates as the end points of the other line segment input. \\
\noindent Give an algoirthm that determines if a given floor plan is ergonomic, or can be mapped so all light fixtures are visible with all light switches, in polynomial $n$ and $m$ time.

\begin{Algorithm}
Determine in $O(mn)$ time if the given floor plan can be ergonomic.
\end{Algorithm}

\begin{proof}
\begin{description}

First, make a bipartite graph of all $n$ switches to all $n$ fixtures. Define edges as all pairs of switches and fixtures that are visible, with capacities of 1. Pairs that are not visible to each other effectively have 0 capacity or no edge.\\
Next, create source and sink nodes, connecting them to all switch and fixture nodes respectively with edges of capacity 1, to simulate supplies and demands of 1 light fixture control connection. \\
This graph is ready for Ford-Fulkerson to be run on it, as discussed in lecture.

\underline{Running Time:} Making initial bipartite graph: $n^2$ for making each node and checking its visibility with each other node, times $m*1$ for checking against each other wall using the given $O(1)$ function. Initial total runtime of $O(n^2 m)$.\\
Creating the source and sink node links can be done at the same time the $n$ switches and fixtures are initialized, so they shouldn't have a significant impact on run time. \\
Ford-Fulkerson runs in $O( \#edges * max_flow)$. Max flow is 1, as all edges are given that value. There are at most $n*(n+2)$ edges, as each fixture-switch pair can share an edge, in addition to the edges connecting the source and sink nodes. These result in an added $O(n^2)$. \\
Total runtime = $O(n^2 m + n^2) = O(n^2 m)$.

\end{description}
\end{proof}


%%%%%%%%%%%%%%%%%%%%%%%%%%
\newpage
\linespread{1.1}
\begin{Question} (6.28) 
\end{Question}

\end{document}  
